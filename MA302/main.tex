\documentclass{article}
\usepackage{exoo}[notoc,fancy]

\newcommand{\sol}{\setlength{\parindent}{0cm}\textbf{\textit{Solution:}}\setlength{\parindent}{1cm} }
\DeclareMathOperator{\Int}{Int}
\DeclareMathOperator{\Bd}{Bd}


\begin{document}

\title{Vector Analysis} % Title
\author{Brady Exoo} % Author
\date{Spring 2026} % Date
\maketitle

\tableofcontents

\eject

\section{Mon January 12}
Midterm on February 25 during class.
\subsection{Introduction}
We will focus on:
\begin{enumerate}
  \item Differential Calculus in $\RR^n$ (derivatives $\nabla f$ are ``best linear approximations''). Basically multivariable calculus, implicit function theorem, inverse function theorem.
  \item Integral calculus in $\RR^n$: on ``nice'' subsets $S \subset \RR^n$, we can define the integral. Also Fubini's theorem (integration on variables one-by-one).
  \item \vocab{Manifolds}: a subset in $\RR^n$ that locally looks like an open set in $\RR^k$ for $0 \leq k \leq n$. Also differentiation on maps between manifolds. To help us with integration, we introduce \vocab{differential forms} (an algebraic object that can be integrated?), something something exterior algebra, exterior derivatives. And Generalized Stoke's Theorem! $\int_{\partial M} \omega = \int_M \dd{\omega}$.
\end{enumerate}

\subsection{Topology in the Reals}
Some review of vocabulary from Analysis.
\begin{definition}
  A \vocab{metric space} is a set $X$ equipped with a distance function $d: X \times X \to \RR$ such that 
  \begin{itemize}
    \item $d(x,y) = d(y,x)$ 
    \item $d(x,y) \geq 0$
    \item $d(x,y) = 0 \iff x=y$
    \item $d(x,z) \leq d(x,y) + d(y,z)$
  \end{itemize}
\end{definition}
We can have metrics such as the usual Euclidean metric $||x-y||$ and the sup metric $\max_{1 \leq i \leq n} |x_i-y_i|$.
We also commonly use the $\epsilon$-ball:
\[B_\epsilon (x) := \{y \in X \mid d(x,y) < \epsilon\}. \]
In the sup metric $d_\infty$, the $\epsilon$-ball for $n=2$ is a square.
\begin{definition}
  A set $U \subseteq X$ is \vocab{open} if $\forall x \in U$, $\exists \epsilon > 0$ s.t. $B_\epsilon(x) \subset U$.

  A set $V \subseteq X$ is \vocab{closed} if $X \setminus V$ is open.
\end{definition}
Here, we define \vocab{neighborhoods} of $x$ to just be an open set containing $x$. We also use
\begin{itemize}
  \item $\Int A = \{x \in A \mid B_\epsilon(x) \subset A \text{ for some $\epsilon > 0$}\}$.
  \item Limit points of $A = \{x \in X \mid (B_\epsilon (x) \setminus \{x\}) \cap A \neq \varnothing \quad \forall \epsilon > 0.\}$
  \item $\overline{A} = A \cup \text{limit points of $A$}$.
  \item $\Bd A = \overline{A} \setminus \Int(A)$.
\end{itemize}
\begin{definition}
 For metric spaces $X$ and $Y$ and $f: X \to Y$, $f$ is continuous at $x_0$ if $\forall \epsilon > 0$, $\exists \delta > 0$ such that $d_y(f(x), f(x_0)) < \epsilon$ whenever $d_x (x, x_0) < \delta$.
\end{definition}
\begin{prop}
 For any dimension $n$,
 \[ B_{\infty, \epsilon/\sqrt{n}}(x) \subset B_{2, \epsilon}(x) \subset B_{\infty, \epsilon}(x). \]
\end{prop}
\proof Homework.

Basically this shows us if $\epsilon \to 0$ in one metric, $\epsilon \to 0$ in the other. And a set in $d_2$ is open/closed iff it is open/closed in $d_\infty$. Similarly, a map $f: \RR^n \to X$ or $X \to \RR^n$ is continuous wrt $d_\infty$ iff it is continuous wrt $d_2$.
\proof Homework.

\begin{prop}
  $f: X \to \RR^n$ with $f = (f_1,\dots,f_n)$. Then $f$ is continuous iff $f_1 \dots f_n$ are all continuous.
\end{prop}
\proof For $\implies$, we say $f$ is continuous in $d_2$, and using the definition of $d_2$ we can put a bound on all of $|f_i(x) - f_i(y)|$. \\ 
For $\impliedby$, we choose $\delta_i$ such that $d_x(x,y) < \delta_i \implies |f_i(x) - f_i(y)| < \frac{\epsilon}{\sqrt{n}}$. Then the math just works out when using the minimum of all the $\delta_i$.

\section{Wed Jan 14}
Differentiation today?
\begin{definition}
  We say $f(x)$ approaches $y_0$ as $x$ approaches $x_0$ if $\forall \epsilon > 0$, $\exists \delta > 0$ s.t. $d_y(f(x), y_0) < \epsilon$ whenever $d_x(x,x_0) < \delta$.
  In this case, we write $\lim_{x \to x_0} f(x) = y_0$.
\end{definition}
Then if $x_0$ is a limit point of $X$, $f$ is continuous at $x_0$ iff $\lim_{x \to x_0} f(x) = f(x_0)$.
Also for an open set $U \subset \RR$ and $f: U \to \RR^n$, $f$ is differentiable if the usual limit exists.
Then we can also write
\[ \lim_{h \to 0} \frac{f(x+h)-[f(x) + f^\prime(x)h]}{h} = 0.\]
\begin{prop}
  Write $f = (f_1,\dots,f_n)$. Then $f$ is differentiable at $x$ iff all $f_1,\dots,f_n$ are differentiable at $x$.
  In this case, $f^\prime(x) = (f_1^\prime, \dots, f_n^\prime)$.
\end{prop}
\proof Homework. Just use $\epsilon$-$\delta$?
\begin{definition}
  Let $U \subset \RR^m$ be open, $f: U \to \RR^n$, $u \in \RR^m \setminus \{0\}$.
  The \vocab{directional derivative} of $f$ along $u$ is defined to be
  \[ f^\prime(x; u) = \lim_{t \to 0}\frac{f(x+tu)-f(x)}{t} \]
\end{definition}
We can regard the directional derivative as a dot product of a matrix and a vector. E.g.\ for $f(x_1,x_2)=x_1 x_2$ and $u=(u_1,u_2)$,
\[ f^\prime(x;u) = \begin{bmatrix} x_2 & x_1 \end{bmatrix} \cdot \begin{bmatrix} u_1 \\ u_2 \end{bmatrix} \]
\begin{prop}
  If $f^\prime(x;u)$ exists and $\lambda \neq 0$, $f^\prime(x;\lambda u) = \lambda f^\prime(x;u)$.
\end{prop}
\proof Substitute $t \to t/\lambda$ in the definition.
\begin{definition}
  For $x \in U$ and $t \in \{1,2,\dots,m\}$, define
  \[ D_i f(x) = \pdv{f}{x_i}(x) := f^\prime(x;e_i) \]
  where $e_i$ is a basis vector in $\RR^n$.
\end{definition}
Now we can extend our definition of the derivative to $\RR^m \to \RR^n$.
\begin{definition}
  Let $U \subset \RR^m$ be open, $f: U \to \RR^n$, $x \in U$.
  We say $f$ is differentiable at $x$ if there exists an $n \times m$ matrix $Df(x)$ such that
  \[ \lim_{h \to 0} \frac{f(x+h) - [f(x) + Df(x) \cdot h]}{\norm{h}} = 0 \]
  where $h \in \RR^m$.
\end{definition}
Now we wish to show $Df(x)$ is unique if it exists. If we have two matrices $A_1$ and $A_2$ that satisfy the property above, we can take the difference, leaving us with
\[ \lim_{h \to 0}(A_1 - A_2) \cdot \frac{h}{\norm{h}} = 0. \]
We can take $h = te_i$ with $t>0$, then $\forall i$,
\[ (A_1-A_2)\cdot e_i = 0 \]
thus all columns of $A_1-A_2$ are $0$, so $A_1-A_2=0$.
\begin{note}
  $Df(x)$ has to be the matrix that sends $f(x+h)-[f(x)+ B \cdot h]$ to $0$ as fast as possible otherwise $B$ would also be a derivative of $f$.
\end{note}
Just like for $\RR$, we have differentiable $\implies$ continuous.
\begin{prop}
  If $f(x) = B \cdot x + b$ where $B$ is an $n \times m$ matrix and $b \in \RR^n$, then $Df(x) = B$.
\end{prop}
\proof Exercise.
\begin{prop}
  If $f$ differentiable at $x$, then $\forall u \in \RR^m \setminus \{0\}$, 
  \[ f^\prime (x;u) = Df(x) \cdot u. \]
\end{prop}
\proof Substitute $h \to tu$.

\end{document}
