\documentclass{article}
\usepackage{exoo}[notoc,fancy]

\newcommand{\sol}{\setlength{\parindent}{0cm}\textbf{\textit{Solution:}}\setlength{\parindent}{1cm} }
\DeclareMathOperator{\Int}{Int}
\DeclareMathOperator{\Bd}{Bd}


\begin{document}

\title{Abstract Algebra} % Title
\author{Brady Exoo} % Author
\date{Spring 2026} % Date
\maketitle

\tableofcontents

\eject

\section{Tues January 13}
One midterm one final. Midterm on groups, final mostly on rings.
\subsection{Introduction}
Motivating question: consider a triangle. How many symmetries are there?

The answer turns out to be $6$ through rotations and reflections, just the permutations of the vertices. We call this set $D_6$, with composition as the operation. Together, this makes a \vocab{group}.
Note that all elements of $D_6$ can be written as compositions of rotations and reflections. We can then say that rotation and reflection \vocab{generate} the group $D_6$.
\begin{definition}
  A \vocab{group} is a set $G$ equipped with a binary operation $\cdot: G \times G \to G$ satisfying the following axioms:
  \begin{enumerate}
  \item Associativity: $(a \cdot b) \cdot c = a \cdot (b \cdot c)$
  \item Identity: $\exists 1 \in G$ s.t. $a \cdot 1 = a$.
  \item Inverse: $\forall a \in G$, $\exists a^{-1}$ s.t. $a^{-1} \cdot a = 1$
  \item Closed: $\forall a,b \in G$, $a \cdot b \in G$. (This is kind of implicit in our definition of $\cdot$).
  \end{enumerate}
\end{definition}
Some other groups/rings:
\begin{itemize}
\item $(\ZZ, +)$ 
\item $(\QQ \setminus 0, \times)$
\item $(\ZZ, +, \times)$: Ring
\item $(\ZZ/12, +)$
\item $S_3$: all the permutations of $3$ elements
\end{itemize}

\section{Thu Jan 15}
Today we will prove some basic properties of groups.
\begin{prop}
  Let $(G, \cdot)$ be a group.
  \begin{enumerate}
  \item The identity is unique.
  \item For any $g \in G$, $g^{-1}$ is unique.
  \item For any $g \in G$, $(g^{-1})^{-1} = g$.
  \item $(a \cdot b)^{-1} = b^{-1} \cdot a^{-1}$.
  \item Associativity works for any $n$
  \end{enumerate}
\end{prop}
\proof Just do the obvious.
\begin{definition}
  The \vocab{order} of an element $g \in G$ is the smallest $n \in \ZZ_{>0}$ such that $g^n = e$.

  The \vocab{order} of a group $G$ is its cardinality $|G|$.
\end{definition}

\begin{definition}
  A subset $S \subseteq G$ \vocab{generates} the group $G$ if 
  \[ \{\text{iterative products of elements of $S$ and their inverses}\} = G.\]
\end{definition}
For example, $\{1\}$ generates $\ZZ$. A word is just an expression like $r^2 s$. A relation in $G$ is a word whose underlying element is the identity. We write $G = <S\mid R>$ if $G$ is generated by $S$ and has relations $R$.
\begin{definition}
  The dihedral group $D_{2n}$ is the set of symmetries of a regular $n$-gon.
\end{definition}
$D_{2n}$ has presentation $<r,s \mid r^n = e, s^2 = e, rs = sr^{-1}>$.
\begin{definition}
  A group is \vocab{Abelian} if $\forall x,y \in G$,
\end{definition}

\end{document}
